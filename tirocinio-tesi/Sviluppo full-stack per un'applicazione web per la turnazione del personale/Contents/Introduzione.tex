\chapter{Introduzione}\label{ch:introduzione}
Tutte le aziende, in particolare quelle di una certa complessità, hanno la necessità di gestire e razionalizzare le risorse e i servizi che erogano ai propri dipendenti, con particolare attenzione a quei servizi che permettono di utilizzare determinati strumenti aziendali e di implementare le comunicazioni ufficiali tra dipendenti e azienda. 
Parte del processo di digitalizzazione delle aziende prevede, tra le altre iniziative, la creazione e l'utilizzo di sistemi automatici, intelligenti e fruibili sia da computer che da dispositivi mobili per l'accesso ai servizi aziendali.

L'azienda presso la quale ho svolto il mio tirocinio è Gruppo SIGLA, un'azienda di più di 100 persone, di cui molte impiegate in via temporanea presso clienti in Italia e all'estero. Gruppo SIGLA ha interesse nel digitalizzare questi processi rapidamente e nel realizzare sistemi che permettano di aiutare altre aziende a fare altrettanto.
In particolare, Gruppo SIGLA vuole implementare una soluzione che permetta di gestire i propri strumenti e servizi aziendali in modo tale da garantirne un accesso rapido ed efficiente per i propri dipendenti.

Il servizio sul quale ho lavorato durante il tirocinio è stato quello di tracciamento e feedback delle attività formative erogate dall'azienda.
Questo sistema si presenta come una applicazione web, composta da elementi client e server, ed è sviluppato tramite l'utilizzo di tecnologie avanzate come \gls{.net} per il back-end e Angular per il front-end. Si prevede, inoltre, la possibilità di integrare servizi di autenticazione e invio di email ai dipendenti.

% Le attività da svolgere durante la tesi sono: Comprensione analisi del servizio (preparata da Gruppo SIGLA) Apprendimento rapido tecnologie back-end e front-end fondamentali Progettazione e implementazione dell'infrastruttura di integrazione catalogo servizi. Sviluppo Servizio di tracciamento delle attività formative. Validazione funzionalità realizzate e test risultati.

Questo documento descrive la mia attività di tirocinio ed è organizzato nei seguenti capitoli.
%
Il Capitolo \ref{ch:Contesto} presenta il contesto di riferimento, in particolare si parla del progetto scelto, delle tecnologie utilizzate (per il back-end e per il front-end), delle metodologie di sviluppo e dei requisiti per portare a termine il progetto.
%
Il Capitolo \ref{ch:prototipo} è dedicato alla descrizione del prototipo e introduce le scelte fatte per la sua implementazione. %Inoltre, si parla della fase di sviluppo e di test del prototipo.
%
Infine, il Capitolo \ref{ch:conclusioni}, presenta conclusioni e sviluppi futuri.
